%% ---------------------------------------------------------------------------
%% proposal.tex
%%
%% Research Proposal, main document.
%%
%% ---------------------------------------------------------------------------
\documentclass[12pt,letterpaper]{article}
\usepackage[english]{babel}     % supports english, but default is
% \usepackage[spanish]{babel}
% include this if you want to import graphics files with /includegraphics

\usepackage{longtable}
\usepackage{ifpdf}
\usepackage[table]{xcolor}

\usepackage{anysize}
\marginsize{2.5cm}{2.5cm}{1cm}{1cm}
\usepackage{textcomp}
\usepackage{url}
\bibliographystyle{unsrt}
\usepackage{graphics}
\usepackage{amssymb}
\usepackage{graphicx}
%\usepackage{slashbox}
\usepackage[latin1]{inputenc}
\usepackage{tikz}
\usepackage{amsmath}
\usepackage{float}
\usetikzlibrary{arrows,positioning}
% Color and strikethrough



\usepackage{color}
\usepackage{soul}

\usepackage{array}
\usepackage{makecell}

\usepackage{sectsty}
\allsectionsfont{\sffamily}

\definecolor{dblue}{RGB}{0,102,153}
\newcommand{\dB}[1]{\textcolor{dblue}{\textbf{#1}}}



\setlength{\parskip}{1em}

% Nombre del Estudiante
\newcommand{\scriptAuthor}{Daniel Moya S�nchez}
 
% T��tulo de la tesis
\newcommand{\scriptTitle}{Progress report \#5 for the project: Design of (ASIPs) for Approximate Computing} 


% Keywords
\newcommand{\scriptKeywords}{key, words, ...}

% Para el PDF (cambiar si se desea otras cosas a lo indicado arriba
\newcommand{\pdfAuthor}{\scriptAuthor}
\newcommand{\pdfTitle}{\scriptTitle} 
\newcommand{\pdfKeywords}{\scriptKeywords}


\tikzset{
    mynode/.style={rectangle,rounded corners,draw=black, top color=white, bottom color=yellow!50,very thick, inner sep=1em, minimum size=3em, text centered},
    myarrow/.style={->, >=latex', shorten >=1pt, thick},
    mylabel/.style={text width=7em, text centered} 
} 

\begin{document}
 
 \graphicspath{{./}{./fig/}}

 %% ---------------------------------------------------------------------------
%% titlepage.tex
%%
%% Title page
%%
%% ---------------------------------------------------------------------------

\thispagestyle{empty} 

\begin{center}

\textsc{\LARGE Instituto Tecnol\'ogico de Costa Rica} \\
\textsc{\Large Computer Engineering Academic Area}

\textsc{\Large Proyecto de Dise\~no en Ingenier\'ia en Computadores}


\par\vspace{20mm}

\includegraphics[scale=0.25]{logoTEC}

\par\vspace*{\fill}

{\LARGE\bf{\textsf{ \Huge \scriptTitle}}}

\par\vspace*{\fill}

%Master Thesis {\sf Proposal} \\ 
%in fulfillment of the requirements for the degree of
%Plan de Proyecto
%Master of Science in Electronics Engineering \\
%Emphasis on Embedded Systems

%\par\vspace{20mm}

\textsc{\Large \scriptAuthor}

\vspace*{\fill}

{\today}

\end{center}
\newpage 
\cleardoublepage  


%  \tableofcontents

 \clearpage

\section{Performed activities} \label{act}

\begin{enumerate}
  
 \item \textbf{Implement the ASIPs in the error tolerant applications found (ID 03)}: The \emph{absv} assembly instruction was implemented for the KNN algorithm.
 This allows computing the following operation in a single cycle:
 \begin{align}
  rd = a > b ? a-b : b-a
 \end{align}
 The operation described in (1) allows the execution of a subtract operation with an absolute result, which is used frequently in the KNN algorithm (the euclidean distance remains the main operation
 in this algorithm too). 
 
 \item \textbf{Implement two small assembly codes for each special instruction developed (ID 10)}: The first manually-written version of the assembly codes were implemented. These consists of a simple
 programs that loads each value of an array (lenght of 100, with all the values explicitly defined) and executes the corresponding operation(\emph{eucl} or \emph{absv}) with the actual special instruction or
 the equivalent with the common assembly instructions.
 
 \item \textbf{Compare execution time, area and power vs error in selected applications (ID 09)}: Comparison at a simulation level was performed, with a lower total cycles found for the 
programs with the special instruction implemented (as expected). With the \emph{eucl} instruction nearly a 10\% of the total cycles reduction was achieved and almost 100\% for the \emph{absv} instruction.

 \end{enumerate}
 

\section{Scope Changes} \label{change}

The total number of approximate applications has officially been changed from 3 to 2
and the scope of the approximations is reduced, eliminating the hardware-level approximation in the ASIPs. The following task was eliminated:

\begin{itemize}
 \item \textbf{Write Test Plan document (ID 07)}: The tests will be performed as the supervisor Jorge Castro requests with a verbal agreement.
\end{itemize}

\noindent The following task was included:

\begin{itemize}
 \item \textbf{Implement two small assembly codes for each special instruction developed (ID 10)}: Since the CoSy compiler did not work properly, a reduced manually-written assembly version
 of the algorithms K-Means and KNN is going to be implemented.
\end{itemize}


\section{Earned Value analysis}

In general, due to the difficulties explained in section \ref{dif}, the project
has a delay of approximately one week considering the new scope, despite having overestimated the cost of tasks (without the configurations problems), which in general have been
done in less time than planned. 

Table \ref{tab:gain} summarizes the gained value analysis. 

\begin{table}[h!]
\begin{center}
\caption{Earned Value} \bigskip
\resizebox{\textwidth}{!}{\begin{tabular}{|c|c|c|c|c|c|c|c|c|c|c|c|c|c|} 
 \hline
\makecell{Activity\\ID}	&Activity	&Budget	&\makecell{\%Planned\\Value}	&PV	&AC	&\makecell{\%Completed\\work}	&EV	
&CPI	&SPI	&\makecell{Initial planned\\date}	&\makecell{Ending\\date}	&\makecell{Initial real\\date}	&\makecell{Real ending} \\ \hline


01	&\makecell{Get to know \\the software\\ platform}	&32	&100\%	&32	&34	&100\%	&$32$
&$0.94$	&1	&Week 1	&Week 3	&Week 1	&Week 7 \\ \hline

02	&\makecell{Find appropiate\\ error-tolerant\\ Applications}	&32	&100\%	&32	&20	&100\%	&32
&1.6	&1	&Week 4	&Week 6	&Week 4	&Week 10 \\ \hline

03	&\makecell{Implement the\\ ASIPs in the\\ error tolerant\\ applications\\ found}	&64	&100\%	&64	&30	&80\%	&51.2
&1.71	&0.8	&Week 7	&Week 12	&Week 9	&- \\ \hline


04	&\makecell{Write Project\\Plan Document}	&8	&100\%	&8	&10	&100\%	&8
&0,8	&1	&Week 1	&Week 2	&Week 1	&Week 3 \\ \hline


05	& \makecell{Write\\Requirements\\Document}	&8	&100\%	&8	&6	&100\%	&8	
&1,33	&1	&\makecell{Week 2}	&\makecell{Week 3}	&\makecell{Week 2}	&\makecell{Week 4} \\ \hline


06	& \makecell{Write Design\\ Document}	&8	&100\%	&8	&7	&100\%	&8
&1.14	&1	&\makecell{Week 3}	&\makecell{Week 4}	&\makecell{Week 3}	&\makecell{Week 5} \\ \hline

09	& \makecell{Compare execution\\ time, area and\\ power vs error\\ in selected \\applications}	&32	&25\%	&8	&5	&30\%	&9.6
&1.92	&1.2	&\makecell{Week 12}	&\makecell{Week 16}	&\makecell{Week 12}	&\makecell{-} \\ \hline

10	& \makecell{Implement two \\small assembly \\codes for each\\ special instruction\\ developed}	&16	&66\%	&10.56	&8	&80\%	&12.8
&1.6	&1.21	&\makecell{Week 11}	&\makecell{Week 13}	&\makecell{Week 11}	&\makecell{-} \\ \hline

& Total	& &	&170.56	&120	&	&161.6
&1.35	&0.95	&	&	&	& \\ \hline
\end{tabular}}
\label{tab:gain}
\end{center}
\end{table}




\section{Difficulties Encountered} \label{dif}

\begin{itemize} 

\item The server nodes which are the ones where the tools like ASIPMeister and Dlxsim work properly were disconnected for more than a week. 
 
 
 \item The normal meeting on tuesday (01/05/2018) was cancelled due to the holiday, which delayed slightly more the progress on the project.
 
\end{itemize}




\section{Hard Skills Required/Acquired}

\begin{itemize}
 \item Knowledge in the following software frameworks has been acquired: ASIPMeister and Dlxsim. 
 
 \item Knowledge in HDL has been reinforced.
\end{itemize}




\section{Soft Skills Required/Acquired}

The following soft skills have been excercised:

\begin{itemize}
 
 \item Self-Motivation: Given that there is no direct round-the-clock supervision, self-motivation has been key in working 
 continously with the ASIPMeister and Dlxsim on the creation of special instructions for the approximate applications.
 
 \item Communication: Remote communication has been performed with Jorge Castro for the guidance of this project.
 
\end{itemize}


\section{Lessons Learned}
There are no lessons learned for the reported period. 


% Referencias del Background y el Related Work
\bibliographystyle{sty/plainurl}
\bibliography{references}



\end{document}

