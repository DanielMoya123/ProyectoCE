%% ---------------------------------------------------------------------------
%% proposal.tex
%%
%% Research Proposal, main document.
%%
%% ---------------------------------------------------------------------------
\documentclass[12pt,letterpaper]{article}
\usepackage[english]{babel}     % supports english, but default is
% \usepackage[spanish]{babel}
% include this if you want to import graphics files with /includegraphics

\usepackage{longtable}
\usepackage{ifpdf}
\usepackage[table]{xcolor}

\usepackage{anysize}
\marginsize{2.5cm}{2.5cm}{1cm}{1cm}
\usepackage{textcomp}
\usepackage{url}
\bibliographystyle{unsrt}
\usepackage{graphics}
\usepackage{amssymb}
\usepackage{graphicx}
%\usepackage{slashbox}
\usepackage[latin1]{inputenc}
\usepackage{tikz}
\usetikzlibrary{arrows,positioning}
% Color and strikethrough



\usepackage{color}
\usepackage{soul}

\usepackage{array}
\usepackage{makecell}

\usepackage{sectsty}
\allsectionsfont{\sffamily}

\definecolor{dblue}{RGB}{0,102,153}
\newcommand{\dB}[1]{\textcolor{dblue}{\textbf{#1}}}



\setlength{\parskip}{1em}

% Nombre del Estudiante
\newcommand{\scriptAuthor}{Daniel Moya S�nchez}

% T��tulo de la tesis
\newcommand{\scriptTitle}{Design Document v1} 


% Keywords
\newcommand{\scriptKeywords}{key, words, ...}

% Para el PDF (cambiar si se desea otras cosas a lo indicado arriba
\newcommand{\pdfAuthor}{\scriptAuthor}
\newcommand{\pdfTitle}{\scriptTitle} 
\newcommand{\pdfKeywords}{\scriptKeywords}


\tikzset{
    mynode/.style={rectangle,rounded corners,draw=black, top color=white, bottom color=yellow!50,very thick, inner sep=1em, minimum size=3em, text centered},
    myarrow/.style={->, >=latex', shorten >=1pt, thick},
    mylabel/.style={text width=7em, text centered} 
} 

\begin{document}
 
 \graphicspath{{./}{./fig/}}

 %% ---------------------------------------------------------------------------
%% titlepage.tex
%%
%% Title page
%%
%% ---------------------------------------------------------------------------

\thispagestyle{empty} 

\begin{center}

\textsc{\LARGE Instituto Tecnol\'ogico de Costa Rica} \\
\textsc{\Large Computer Engineering Academic Area}

\textsc{\Large Proyecto de Dise\~no en Ingenier\'ia en Computadores}


\par\vspace{20mm}

\includegraphics[scale=0.25]{logoTEC}

\par\vspace*{\fill}

{\LARGE\bf{\textsf{ \Huge \scriptTitle}}}

\par\vspace*{\fill}

%Master Thesis {\sf Proposal} \\ 
%in fulfillment of the requirements for the degree of
%Plan de Proyecto
%Master of Science in Electronics Engineering \\
%Emphasis on Embedded Systems

%\par\vspace{20mm}

\textsc{\Large \scriptAuthor}

\vspace*{\fill}

{\today}

\end{center}
\newpage 
\cleardoublepage  


%  \tableofcontents

 \clearpage
 
 \begin{table}[h!]
\begin{center}
\caption{Revision History}
\begin{tabular}{ | c | c | c | c | } 
 \hline
\textbf{Date}	&\textbf{Version}	&\textbf{Description}	&\textbf{Author} \\ \hline
02 March 2018	&1.0			&\makecell{Design Document for Design of \\  (ASIPs) for Approximate Computing}	&Daniel Moya \\ \hline
\end{tabular}
\label{tab:act}
\end{center}
\end{table}



\section{Introduction}



\subsection{Purpose}
The primary purpose of this document is to present a detailed description of the design elements
of an ASIP.



\subsection{Scope}
This project is going to be implemented... 

Future users will be able to...

This project will be implemented between ...



%Systems services and users
\subsection{Context}

\subsection{Summary}






% Referencias del Background y el Related Work
\bibliographystyle{sty/plainurl}
\bibliography{references}


\section{Glossary}


% Composition and modular assembly of
% systems in terms of subsystems and
% (pluggable) components, buy vs. build,
% reuse of components
\section{Composition}



% Static structure (classes, interfaces, and
% their relationships)
% Reuse of types and implementations
% (classes, data types)
\section{Logical}


% Interconnection, sharing, and
% parameterization
\section{Dependency}


%Information with data
% distribution overlay and physical
% volumetric overlay
% Persistent information
\section{Information}

% Reuse of patterns and available
% Framework template
\section{Patterns}


% Service definition, service access
\section{Interfaces}

%Service definition, service access
\subsection{User interface}

% Internal constituents and organization of
% design subjects, components and classes
\section{Structure}

\section{Interaction}

\section{State dynamics}

\section{Algorithm}

\section{Resources}





\end{document}

