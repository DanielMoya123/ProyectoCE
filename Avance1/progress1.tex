%% ---------------------------------------------------------------------------
%% proposal.tex
%%
%% Research Proposal, main document.
%%
%% ---------------------------------------------------------------------------
\documentclass[12pt,letterpaper]{article}
\usepackage[english]{babel}     % supports english, but default is
% \usepackage[spanish]{babel}
% include this if you want to import graphics files with /includegraphics

\usepackage{longtable}
\usepackage{ifpdf}
\usepackage[table]{xcolor}

\usepackage{anysize}
\marginsize{2.5cm}{2.5cm}{1cm}{1cm}
\usepackage{textcomp}
\usepackage{url}
\bibliographystyle{unsrt}
\usepackage{graphics}
\usepackage{amssymb}
\usepackage{graphicx}
%\usepackage{slashbox}
\usepackage[latin1]{inputenc}
\usepackage{tikz}
\usetikzlibrary{arrows,positioning}
% Color and strikethrough



\usepackage{color}
\usepackage{soul}

\usepackage{array}
\usepackage{makecell}

\usepackage{sectsty}
\allsectionsfont{\sffamily}

\definecolor{dblue}{RGB}{0,102,153}
\newcommand{\dB}[1]{\textcolor{dblue}{\textbf{#1}}}



\setlength{\parskip}{1em}

% Nombre del Estudiante
\newcommand{\scriptAuthor}{Daniel Moya S�nchez}
 
% T��tulo de la tesis
\newcommand{\scriptTitle}{Progress report \#1 for the project: Design of (ASIPs) for Approximate Computing} 


% Keywords
\newcommand{\scriptKeywords}{key, words, ...}

% Para el PDF (cambiar si se desea otras cosas a lo indicado arriba
\newcommand{\pdfAuthor}{\scriptAuthor}
\newcommand{\pdfTitle}{\scriptTitle} 
\newcommand{\pdfKeywords}{\scriptKeywords}


\tikzset{
    mynode/.style={rectangle,rounded corners,draw=black, top color=white, bottom color=yellow!50,very thick, inner sep=1em, minimum size=3em, text centered},
    myarrow/.style={->, >=latex', shorten >=1pt, thick},
    mylabel/.style={text width=7em, text centered} 
} 

\begin{document}
 
 \graphicspath{{./}{./fig/}}

 %% ---------------------------------------------------------------------------
%% titlepage.tex
%%
%% Title page
%%
%% ---------------------------------------------------------------------------

\thispagestyle{empty} 

\begin{center}

\textsc{\LARGE Instituto Tecnol\'ogico de Costa Rica} \\
\textsc{\Large Computer Engineering Academic Area}

\textsc{\Large Proyecto de Dise\~no en Ingenier\'ia en Computadores}


\par\vspace{20mm}

\includegraphics[scale=0.25]{logoTEC}

\par\vspace*{\fill}

{\LARGE\bf{\textsf{ \Huge \scriptTitle}}}

\par\vspace*{\fill}

%Master Thesis {\sf Proposal} \\ 
%in fulfillment of the requirements for the degree of
%Plan de Proyecto
%Master of Science in Electronics Engineering \\
%Emphasis on Embedded Systems

%\par\vspace{20mm}

\textsc{\Large \scriptAuthor}

\vspace*{\fill}

{\today}

\end{center}
\newpage 
\cleardoublepage  


%  \tableofcontents

 \clearpage

\section{Performed activities} \label{act}

\begin{enumerate}
 \item \textbf{Get to know the software platform}: Several laboratory scripts have been followed to get to know the software tools from which the ASIPs are going
 to be built. These laboratory scripts consist of several excercises and questions (an answer sheet is available for comparison) to get a student through all the necessary 
 knowledge for building ASIPs, from the basics of an assembly program to an audio application which needs to be optimized. However, this activity was affected
 by server errors like permissions and general configuration of the environment. The
corresponding laboratory sessions made for this task are not completely finished, it is nevertheless expected to work on the sessions on parallel to the activity ID
\emph{02}, because they do not depend on each other, and the main concepts have been already learned from the currently done laboratory sessions. 
 
 \item \textbf{Find appropiate error-tolerant applications}: General possible aplications have been discussed.
 
 \item \textbf{Redact Project Plan document}: The project plan document was revised and corrected according to the professor's observations. 
 
 \item \textbf{Redact Requirements document}: The requirements document was redacted and sent to the professor for his possible pre-review.
 
 \item \textbf{Redact Design document}: The design document was redacted and sent to the professor for his possible pre-review.

 \end{enumerate}
 

\section{Change of scope} \label{change}

Since the project is still on its initial phase, no change of scope has been made. 

\section{Gained value analysis}

Table \ref{tab:gain} summarizes the gained value analysis. 

\begin{table}[h!]
\begin{center}
\caption{Revision History}
\resizebox{\textwidth}{!}{\begin{tabular}{|c|c|c|c|c|c|c|c|c|c|c|c|c|c|} 
 \hline
\makecell{Activity\\ID}	&Activity	&Budget	&\makecell{\%Planned\\Value}	&PV	&AC	&\makecell{\%Completed\\work}	&EV	
&CPI	&SPI	&\makecell{Initial planned\\date}	&\makecell{Ending\\date}	&\makecell{Initial real\\date}	&\makecell{Real ending} \\ \hline


01	&\makecell{Get to\\ know the\\ software\\ platform}	&32	&100\%	&32	&25	&80\%	&25,6	
&1,02	&0.8	&Week 1	&Week 3	&Week 1	&- \\ \hline

02	&\makecell{Find appropiate\\ error-tolerant\\ applications}	&32	&33\%	&10,56	&2	&10\%	&3,2	
&1,6	&0.3	&Week 4	&Week 7	&Week 4	&- \\ \hline

04	&\makecell{Redact Project\\Plan document}	&8	&100\%	&8	&10	&100\%	&8
&0,8	&1	&Week 1	&Week 2	&Week 1	&Week 3 \\ \hline


05	& \makecell{Redact\\Requirements\\document}	&8	&100\%	&8	&6	&100\%	&8	
&1,33	&1	&\makecell{Week 2}	&\makecell{Week 3}	&\makecell{Week 2}	&\makecell{Week 4} \\ \hline



06	& \makecell{Redact Design\\ document}	&8	&100\%	&8	&7	&90\%	&7,2	
&1,14	&0,9	&\makecell{Week 3}	&\makecell{Week 4}	&\makecell{Week 3}	&\makecell{-} \\ \hline
\end{tabular}}
\label{tab:gain}
\end{center}
\end{table}




\section{Encountered difficulties}

As explained in section \ref{act}, several issues have been encountered when executing the laboratory sessions for the activity \emph{01}. The
time-zone difference between Costa Rica and Germany has slowed down the solutions to these problems, because no matter how small
a problem is (in terms of time required to solve it) a solution comes, at least until the next day. 

\section{Hard skills required/acquired}

Knowledge in the following software frameworks has been acquired: ASIPMeister, Dlxsim and ModelSim. 


\section{Soft skills required/acquired}

The following soft skills have been excercised:

\begin{itemize}
 \item Communication: Weekly remote communication has been performed with Jorge Castro for the guidance of this project, and
 with Sajjad Hussain to request technical aid in the server. With both, swift communication was achieved, each topic that
 was talked was resolved or clarified in very few messages (one or two at most).
 
 \item Self-Motivation: Given that there is no direct round-the-clock supervision, self-motivation has been key in working 
 continously in the laboratory sessions. 
 
 \item Problem Solving: The laboratory sessions provide several challenges given the theoretical aspects of assembly instructions and
 processor structure, which need to be addressed and later compared with a given solution.
\end{itemize}


\section{Learned lessons}

\begin{enumerate}
 \item Special care has to be taken when working with people around the world. Time-zones restrict the options of when a meeting can
 happen and limit how fast a response can be obtained.
 
 \item When customizing ASIP configurations, special directory structure is needed for the corresponding scripts and software platform in general
 to work properly. This structure also helps make development of the laboratory sessions more efficient.  
\end{enumerate}



% Referencias del Background y el Related Work
\bibliographystyle{sty/plainurl}
\bibliography{references}



\end{document}

