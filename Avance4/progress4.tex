%% ---------------------------------------------------------------------------
%% proposal.tex
%%
%% Research Proposal, main document.
%%
%% ---------------------------------------------------------------------------
\documentclass[12pt,letterpaper]{article}
\usepackage[english]{babel}     % supports english, but default is
% \usepackage[spanish]{babel}
% include this if you want to import graphics files with /includegraphics

\usepackage{longtable}
\usepackage{ifpdf}
\usepackage[table]{xcolor}

\usepackage{anysize}
\marginsize{2.5cm}{2.5cm}{1cm}{1cm}
\usepackage{textcomp}
\usepackage{url}
\bibliographystyle{unsrt}
\usepackage{graphics}
\usepackage{amssymb}
\usepackage{graphicx}
%\usepackage{slashbox}
\usepackage[latin1]{inputenc}
\usepackage{tikz}
\usepackage{amsmath}
\usepackage{float}
\usetikzlibrary{arrows,positioning}
% Color and strikethrough



\usepackage{color}
\usepackage{soul}

\usepackage{array}
\usepackage{makecell}

\usepackage{sectsty}
\allsectionsfont{\sffamily}

\definecolor{dblue}{RGB}{0,102,153}
\newcommand{\dB}[1]{\textcolor{dblue}{\textbf{#1}}}



\setlength{\parskip}{1em}

% Nombre del Estudiante
\newcommand{\scriptAuthor}{Daniel Moya S�nchez}
 
% T��tulo de la tesis
\newcommand{\scriptTitle}{Progress report \#4 for the project: Design of (ASIPs) for Approximate Computing} 


% Keywords
\newcommand{\scriptKeywords}{key, words, ...}

% Para el PDF (cambiar si se desea otras cosas a lo indicado arriba
\newcommand{\pdfAuthor}{\scriptAuthor}
\newcommand{\pdfTitle}{\scriptTitle} 
\newcommand{\pdfKeywords}{\scriptKeywords}


\tikzset{
    mynode/.style={rectangle,rounded corners,draw=black, top color=white, bottom color=yellow!50,very thick, inner sep=1em, minimum size=3em, text centered},
    myarrow/.style={->, >=latex', shorten >=1pt, thick},
    mylabel/.style={text width=7em, text centered} 
} 

\begin{document}
 
 \graphicspath{{./}{./fig/}}

 %% ---------------------------------------------------------------------------
%% titlepage.tex
%%
%% Title page
%%
%% ---------------------------------------------------------------------------

\thispagestyle{empty} 

\begin{center}

\textsc{\LARGE Instituto Tecnol\'ogico de Costa Rica} \\
\textsc{\Large Computer Engineering Academic Area}

\textsc{\Large Proyecto de Dise\~no en Ingenier\'ia en Computadores}


\par\vspace{20mm}

\includegraphics[scale=0.25]{logoTEC}

\par\vspace*{\fill}

{\LARGE\bf{\textsf{ \Huge \scriptTitle}}}

\par\vspace*{\fill}

%Master Thesis {\sf Proposal} \\ 
%in fulfillment of the requirements for the degree of
%Plan de Proyecto
%Master of Science in Electronics Engineering \\
%Emphasis on Embedded Systems

%\par\vspace{20mm}

\textsc{\Large \scriptAuthor}

\vspace*{\fill}

{\today}

\end{center}
\newpage 
\cleardoublepage  


%  \tableofcontents

 \clearpage

\section{Performed activities} \label{act}

\begin{enumerate}
  
 \item \textbf{Implement the ASIPs in the error tolerant applications found (ID 03)}: The \emph{eucl} assembly instruction was made for the K-Means algorithm.
 This allows to compute the following operation in a single cycle:
 \begin{align}
  rd = (rs0 - rs1)^2
 \end{align}
 The operation described in (1) allows the execution of the euclidean distance, which is a key calcution in the K-Means algorithm. No other special instructions
 have been made because it was not possible to have a meeting in the last week with the supervisor Jorge Castro. 
 
 \item \textbf{Find appropiate error-tolerant applications (ID 02)}: The scope of this task is expected to be limited due to the actual project delay. No progress was
 achieved for this task because getting into an agreement about the scope limitation with the supervisor Jorge Castro is expected. 


 \end{enumerate}
 

\section{Scope Changes} \label{change}

No scope changes have been officially made, because there was no meeting with the supervisor Jorge Castro
on week 10, however, scope changes are expected, which include the limitation of the total number of approximate applications
and the scope of the approximations, giving the limitations explained on section \ref{dif}. This changes could solve the actual delay of the project. 

\section{Earned Value analysis}

The tasks ID with 02 and 03 have been delayed because of the problems reported in section \ref{dif} and in previous reports. In general, the project
has a delay of approximately two weeks, despite having overestimated the cost of tasks (without the configurations problems), which have been
done in less time than planned. 

Table \ref{tab:gain} summarizes the gained value analysis. 

\begin{table}[h!]
\begin{center}
\caption{Earned Value} \bigskip
\resizebox{\textwidth}{!}{\begin{tabular}{|c|c|c|c|c|c|c|c|c|c|c|c|c|c|} 
 \hline
\makecell{Activity\\ID}	&Activity	&Budget	&\makecell{\%Planned\\Value}	&PV	&AC	&\makecell{\%Completed\\work}	&EV	
&CPI	&SPI	&\makecell{Initial planned\\date}	&\makecell{Ending\\date}	&\makecell{Initial real\\date}	&\makecell{Real ending} \\ \hline


01	&\makecell{Get to\\ know the\\ software\\ platform}	&32	&100\%	&32	&34	&100\%	&$32$
&$0.94$	&1	&Week 1	&Week 3	&Week 1	&Week 7 \\ \hline

02	&\makecell{Find appropiate\\ error-tolerant\\ Applications}	&32	&100\%	&32	&20	&85\%	&27.2
&1.36	&0.85	&Week 4	&Week 6	&Week 4	&- \\ \hline

03	&\makecell{Implement the\\ ASIPs in the\\ error tolerant\\ applications\\ found}	&64	&66.7\%	&42.7	&15	&40\%	&25.6
&1.71	&0.60	&Week 7	&Week 12	&Week 9	&- \\ \hline


04	&\makecell{Write Project\\Plan Document}	&8	&100\%	&8	&10	&100\%	&8
&0,8	&1	&Week 1	&Week 2	&Week 1	&Week 3 \\ \hline


05	& \makecell{Write\\Requirements\\Document}	&8	&100\%	&8	&6	&100\%	&8	
&1,33	&1	&\makecell{Week 2}	&\makecell{Week 3}	&\makecell{Week 2}	&\makecell{Week 4} \\ \hline


06	& \makecell{Write Design\\ Document}	&8	&100\%	&8	&7	&100\%	&8
&1.14	&1	&\makecell{Week 3}	&\makecell{Week 4}	&\makecell{Week 3}	&\makecell{Week 5} \\ \hline

& Total	& &	&130.7	&92	&	&108.8
&1.18	&0.83	&	&	&	& \\ \hline
\end{tabular}}
\label{tab:gain}
\end{center}
\end{table}




\section{Difficulties Encountered} \label{dif}

\begin{itemize} 

\item No official response has been received regarding the errors of the CoSy compiler, so in the meantime, a manually-wrriten assembly program has been made,
containing only
the special assembly instruction for testing purposes. If the errors of the CoSy compiler are not dealt with, the final result will contain only a manually-wrriten version
made in assembly, which, due to schedule limitations, will not be the full program (only the calculation process). 
 
 
 \item Approximation at a hardware level could not be done for the K-Means algorithm, since the framework ASIPMeister handles a ``Flexible Hardware Model'' (FHM),
 which are files with xml and perl code, which were not expected (the understanding of that code is still needed), in their place, HDL code was expected.
 
 
 \item The supervisor Jorge Castro was quite busy for the reported period, which delayed the next possible special instruction's selection and the 
 analysis of the area and power results obtained from the usage of the first special instruction. 
\end{itemize}




\section{Hard Skills Required/Acquired}

\begin{itemize}
 \item Knowledge in the following software frameworks has been acquired: ASIPMeister and Dlxsim. 
 
\end{itemize}




\section{Soft Skills Required/Acquired}

The following soft skills have been excercised:

\begin{itemize}
 
 \item Self-Motivation: Given that there is no direct round-the-clock supervision, self-motivation has been key in working 
 continously with the ASIPMeister and Dlxsim on the creation of special instructions for the approximate applications.
 
\end{itemize}


\section{Lessons Learned}
There are no lessons learned for the reported period. 


% Referencias del Background y el Related Work
\bibliographystyle{sty/plainurl}
\bibliography{references}



\end{document}

