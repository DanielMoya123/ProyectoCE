%% ---------------------------------------------------------------------------
%% proposal.tex
%%
%% Research Proposal, main document.
%%
%% ---------------------------------------------------------------------------
\documentclass[12pt,letterpaper]{article}
\usepackage[english]{babel}     % supports english, but default is
% \usepackage[spanish]{babel}
% include this if you want to import graphics files with /includegraphics

\usepackage{longtable}
\usepackage{ifpdf}
\usepackage[table]{xcolor}

\usepackage{anysize}
\marginsize{2.5cm}{2.5cm}{1cm}{1cm}
\usepackage{textcomp}
\usepackage{url}
\bibliographystyle{unsrt}
\usepackage{graphics}
\usepackage{amssymb}
\usepackage{graphicx}
%\usepackage{slashbox}
\usepackage[latin1]{inputenc}
\usepackage{tikz}
\usetikzlibrary{arrows,positioning}
% Color and strikethrough



\usepackage{color}
\usepackage{soul}

\usepackage{array}
\usepackage{makecell}

\usepackage{sectsty}
\allsectionsfont{\sffamily}

\definecolor{dblue}{RGB}{0,102,153}
\newcommand{\dB}[1]{\textcolor{dblue}{\textbf{#1}}}



\setlength{\parskip}{1em}

% Nombre del Estudiante
\newcommand{\scriptAuthor}{Daniel Moya S�nchez}

% T��tulo de la tesis
\newcommand{\scriptTitle}{Requirements Specification} 


% Keywords
\newcommand{\scriptKeywords}{key, words, ...}

% Para el PDF (cambiar si se desea otras cosas a lo indicado arriba
\newcommand{\pdfAuthor}{\scriptAuthor}
\newcommand{\pdfTitle}{\scriptTitle} 
\newcommand{\pdfKeywords}{\scriptKeywords}


\tikzset{
    mynode/.style={rectangle,rounded corners,draw=black, top color=white, bottom color=yellow!50,very thick, inner sep=1em, minimum size=3em, text centered},
    myarrow/.style={->, >=latex', shorten >=1pt, thick},
    mylabel/.style={text width=7em, text centered} 
} 

\begin{document}
 
 \graphicspath{{./}{./fig/}}

 %% ---------------------------------------------------------------------------
%% titlepage.tex
%%
%% Title page
%%
%% ---------------------------------------------------------------------------

\thispagestyle{empty} 

\begin{center}

\textsc{\LARGE Instituto Tecnol\'ogico de Costa Rica} \\
\textsc{\Large Computer Engineering Academic Area}

\textsc{\Large Proyecto de Dise\~no en Ingenier\'ia en Computadores}


\par\vspace{20mm}

\includegraphics[scale=0.25]{logoTEC}

\par\vspace*{\fill}

{\LARGE\bf{\textsf{ \Huge \scriptTitle}}}

\par\vspace*{\fill}

%Master Thesis {\sf Proposal} \\ 
%in fulfillment of the requirements for the degree of
%Plan de Proyecto
%Master of Science in Electronics Engineering \\
%Emphasis on Embedded Systems

%\par\vspace{20mm}

\textsc{\Large \scriptAuthor}

\vspace*{\fill}

{\today}

\end{center}
\newpage 
\cleardoublepage  


%  \tableofcontents

 \clearpage

\section{Introduction}


% Delineate the purpose of the software to be specified.
\subsection{Purpose}
The purpose of this research is to evaluate the design of Application-Specific Instruction Set Processors (ASIPs) for 
error-tolerant applications. The ASIPs are meant to execute with improved performance (for instance, 
energy consumption or execution time would be lower) compared to a General Purpose Proccesor (GPP) but
also have greater flexibility than ASICs.

% Describe the scope of the software under consideration by
% a) Identifying the software product(s) to be produced by name (e.g., Host DBMS, Report Generator, etc.);
% b) Explaining what the software product(s) will do;
% c) Describing the application of the software being specified, including relevant benefits, objectives, and
% goals;
% d) Being consistent with similar statements in higher-level specifications (e.g., the system requirements
% specification), if they exist.
\subsection{Scope}
This research concerns both application selection and application optmization. For this, ASIP configurations using specific approximated instructions for the 
 selected applications are to be delivered. Furthermore, each ASIP configuration will be described
 by a set of parameters that the final system will possess, such as energy efficiency, area, execution time, and output error.
 This project is expected to help make approximated computing a more widespread tendency and generate a strong base knowledge for future projects
 where there is freedom to choose the parameters of the hardware running a certain type of an application in terms of resource consumption and accepted error. 




\subsection{Product overview}


% Define the system's relationship to other related products.
% If the product is an element of a larger system, then relate the requirements of that larger system to the
% functionality of the product covered by the SRS.
% If the product is an element of a larger system, then identify the interfaces between the product covered by the
% SRS and the larger system of which the product is an element.
% A block diagram showing the major elements of the larger system, interconnections, and external interfaces
% can be helpful.
% Describe how the software operates within the following constraints:
% a) System interfaces;
% b) User interfaces;
% c) Hardware interfaces;
% d) Software interfaces;
\subsubsection{Product perspective}
The generated result will consist of studied and selected applications where one of its sections or just an instruction
is replaced by an approximated version, to improve execution time, area, or power consumption while
maintaining an acceptable error threshold. This research is a stand-alone project in the sense that the ASIP configurations
made will be specific to a selected application and will not operate with other interfaces. This means that the configurations
developed will only be constrained by each specific application's inherent constraints or limitations, for example, which section is error-tolerant or
which parameter is the most critical. 




% Provide a summary of the major functions that the software will perform. For example, an SRS for an
% accounting program may use this part to address customer account maintenance, customer statement, and
% invoice preparation without mentioning the vast amount of detail that each of those functions requires.
% Sometimes the function summary that is necessary for this part can be taken directly from the section of the
% higher-level specification (if one exists) that allocates particular functions to the software product.
% Note that for the sake of clarity
% a) The product functions should be organized in a way that makes the list of functions understandable to the
% acquirer or to anyone else reading the document for the first time.
% b) Textual or graphical methods can be used to show the different functions and their relationships. Such a
% diagram is not intended to show a design of a product, but simply shows the logical relationships among
% variables.
\subsubsection{Product functions}
With the research on ASIPs, the approximate computing paradigm will be expanded, which will 
enable more personalized applications to balance flexibility and performance to have a 
good trade-off between those variables.

The ASIPs that will be integrated to an specific application will have the following characteristics:

\begin{itemize}
 \item Adjustable error threshold: The error threshold can vary according to a specific ASIP configuration.
 
 \item Adjustable resource consumption: The resource consumption can vary according to a specific ASIP configuration.
 
 \item Scalable: Regardless of the amount of hardware used, the ASIP parameters are expected to remain roughly the
 same.
\end{itemize}



% Describe those general characteristics of the intended groups of users of the product including characteristics
% that may influence usability, such as educational level, experience, disabilities, and technical expertise. This
% description should not state specific requirements, but rather should state the reasons why certain specific
% requirements are later specified in specific requirements in subclause 9.5.9.
\subsubsection{User characteristics}
Since approximated computing is still in its infancy, a lot of research and testing is still needed, 
so the users of the developed ASIPs are the same research groups of which this 
project is a part of. The research group members possess, in general, these characteristics:

\begin{itemize}
 \item Technical knowledge: It is expected that users have an in-depth knowledge of the specific application
 that is being optimized, as well as a fair amount of knowledge of hardware architecture design. 
 
 \item Access to specialized tools: Users are expected to have a set of tools that allow them to synthetize
 and simulate a hardware architecture, as well as a hardware board to implement a desired design. 
\end{itemize}



% Provide a general description of any other items that will limit the supplier's options, including
% a) Regulatory policies;
% b) Hardware limitations (e.g., signal timing requirements);
% c) Interfaces to other applications;
% d) Parallel operation;
% e) Audit functions;
% f) Control functions;
% g) Higher-order language requirements;
% h) Signal handshake protocols (e.g., XON-XOFF, ACK-NACK);
% i) Quality requirements (e.g., reliability)
% j) Criticality of the application;
% k) Safety and security considerations.
% l) Physical/mental considerations
\subsubsection{Limitations}
The following limitations are considered for the ASIPs developed:
\begin{itemize}
 \item Performance gain: The gain in any of the four parameters (energy efficiency, area, execution time and output error) can never be
 greater than the theoretical possible gain given by the Amdahl equation, which is determined by the percentage of the application
 that is being optimized. 
 
 \item Development time: The 16 weeks of the institutional calendar of the ITCR limit the scope of the project to the configurations that can
 be developed in such a short amount of time.
\end{itemize}

The resources needed for the project (e.g hardware platform) are not considered as limitations since they are already given by the research group
and the academic institutions. 

\subsection{Definitions, acronyms and abbreviations}
Table \ref{tab:def} contains the specific terms used in this document.


\begin{table}[h!]
\begin{center}
\caption{Definitions}
\begin{tabular}{|c|c|} 
 \hline
 Term & Definition \\ \hline
 ASIP & \makecell{
 Application Specific Instruction Set Processors. this means that, although the \\
 processor can execute a wide range of applications, it is optimized for a specific \\
 one, in which it can execute with improved performance (for instance, energy \\
 consumption or execution time would be lower) compared to a General Purpose \\
 Proccesor (GPP).}  \\ \hline
\end{tabular}
\label{tab:def}
\end{center}
\end{table}


% Referencias del Background y el Related Work
\bibliographystyle{sty/plainurl}
\bibliography{references}


\section{Specific Requirements}



% Define all inputs into and outputs from the software system. The description should complement the interface
% descriptions in 9.5.3.3.1 through 9.5.3.3.5, and should not repeat information there.
% Each interface defined should include the following content:
% a) Name of item;
% b) Description of purpose;
% c) Source of input or destination of output;
% d) Valid range, accuracy, and/or tolerance;
% e) Units of measure;
% f) Timing;
% g) Relationships to other inputs/outputs;
% h) Screen formats/organization;
% i) Window formats/organization;
% j) Data formats;
% k) Command formats;
% l) Endmessages.
\subsection{External interfaces}
 As explained in section ``Product perspective'', the entire optimizaded application
will not have external interfaces.




% Define the fundamental actions that have to take place in the software in accepting and processing the inputs
% and in processing and generating the outputs, including
% a) Validity checks on the inputs
% b) Exact sequence of operations
% c) Responses to abnormal situations, including
%   1) Overflow
%   2) Communication facilities
%   3) Error handling and recovery
% d) Effect of parameters
% e) Relationship of outputs to inputs, including
%   1) Input/output sequences
%   2) Formulas for input to output conversion
% It may be appropriate to partition the functional requirements into subfunctions or subprocesses. This does
% not imply that the software design will also be partitioned that way.
\subsection{Functions}

\subsubsection{Functional requirement 1.1}
\textbf{ID: FR-1} \\
Title: ASIP optimization \\
Description: The generated knowledge shall provide the user with all the details of a
specific optimization in a given application, so that the user can take this as a reference.


% Define usability (quality in use) requirements. Usability requirements and objectives for the software system
% include measurable effectiveness, efficiency, and satisfaction criteria in specific contexts of use.
\subsection{Usability requirements}
As explained in section ``User characteristics'', the users are expected to be researchers, for this only few 
usability requirements are considered:


\subsubsection{Usability requirement 1.1}
\textbf{ID: UR-1} \\
Title: Knowledge documentation \\
Description: The research shall present the user a proper documentation for each selected application (application domain, i.e. the 
application without any modification) and the specific optimizations performed, which include test cases, gain in 
energy efficiency, area, and power consumption, as well as the output error. 





% Specify both the static and the dynamic numerical requirements placed on the software or on human
% interaction with the software as a whole.
% Static numerical requirements may include the following:
% a) The number of terminals to be supported;
% b) The number of simultaneous users to be supported;
% c) Amount and type of information to be handled.
% Static numerical requirements are sometimes identified under a separate section entitled Capacity.
% Dynamic numerical requirements may include, for example, the numbers of transactions and tasks and the
% amount of data to be processed within certain time periods for both normal and peak workload conditions.
% The performance requirements should be stated in measurable terms.
% For example,
% 95 % of the transactions shall be processed in less than 1 second.
% rather than,
% An operator shall not have to wait for the transaction to complete.
\subsection{Performance requirements}


queremos identificar qu� tanto se puede optimizar el rendimiento de un programa tomando en cuenta 
las variables de tiempo de ejecuci�n, potencia, �rea y error aceptable


%Specify the logical requirements for any information that is to be placed into a database, including:
% a) Types of information used by various functions;
% b) Frequency of use;
% c) Accessing capabilities;
% d) Data entities and their relationships;
% e) Integrity constraints;
% f) Data retention requirements.
\subsection{Logical database requirements}
No database is used for this project, hence, no logical database requirements are presented. 


% Specify constraints on the system design imposed by external standards, regulatory requirements, or project
% limitations.
\subsection{Design constraints}
Design contraints are understood as aspects which are fixed for the project. Since this is a research project, minimun design contraints are presented:

\begin{itemize}
 \item Hardware development board: The project is going to use a Xilinx Virtex-V board, since
 it corresponds to the hardware that uses the development framework. 
 
 \item Development framework:
 
 \item 	Software architecture:
 
 \item Programming language:
\end{itemize}




% Specify the required attributes of the software product. The following is a partial list of examples:
% a) Reliability - Specify the factors required to establish the required reliability of the software system at time
% of delivery.
% b) Availability - Specify the factors required to guarantee a defined availability level for the entire system
% such as checkpoint, recovery, and restart.
% c) Security - Specify the requirements to protect the software from accidental or malicious access, use
% modification, destruction, or disclosure. Specific requirements in this area could include the need to:
%   1) Utilize certain cryptographic techniques;
%   2) Keep specific log or history data sets;
%   3) Assign certain functions to different modules;
%   4) Restrict communications between some areas of the program;
%   5) Check data integrity for critical variables;
%   6) Assure data privacy.
% d) Maintainability - Specify attributes of software that relate to the ease of maintenance of the software itself.
% These may include requirements for certain modularity, interfaces, or complexity limitation. Requirements
% should not be placed here just because they are thought to be good design practices.
% e) Portability - Specify attributes of software that relate to the ease of porting the software to other host
% machines and/or operating systems, including:
%   1) Percentage of elements with host-dependent code;
%   2) Percentage of code that is host dependent;
%   3) Use of a proven portable language;
%   4) Use of a particular compiler or language subset;
%   5) Use of a particular operating system.
\subsection{Software system attributes}


% The SRS should contain additional supporting information including
% a) Sample input/output formats, descriptions of cost analysis studies, or results of user surveys;
% b) Supporting or background information that can help the readers of the SRS;
% c) A description of the problems to be solved by the software;
% d) Special packaging instructions for the code and the media to meet security, export, initial loading, or other
% requirements.
% The SRS should explicitly state whether or not these information items are to be considered part of the
% requirements.
\subsection{Supporting information}


\section{Verification}

\subsection{External interfaces}

\subsection{Functions}

Xilinx, Mentor y Synopsys

\subsection{Usability requirements}

\subsection{Performance requirements}

\subsection{Logical database requirements}

\subsection{Design constraints}

\subsection{Software system attributes}

\subsection{Supporting information}


\section{Appendices}


% List each of the factors that affect the requirements stated in the SRS. These factors are not design
% constraints on the software but any changes to these factors can affect the requirements in the SRS. For
% example, an assumption may be that a specific operating system will be available on the hardware designated
% for the software product. If, in fact, the operating system is not available, the SRS would then have to change
% accordingly.
\subsection{Assumptions and dependencies}
This project is based on the assumption that, given research and articles in the area of approximate computation,
it is feasible to introduce functions or approximate blocks in a given application in order to introduce error and
at the same time have an improvement in resources consumption, so that both variables can be manipulated.


%\subsection{Traceability Matrix}

\end{document}

